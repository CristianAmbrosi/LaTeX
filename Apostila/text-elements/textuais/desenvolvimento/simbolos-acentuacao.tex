% SIMBOLOS E ACENTUAÇÃO-------------------------------------------------------------------

\chapter{SIMBOLOS E ACENTUAÇÃO}

\TeX{} usa ASCII por padrão. Mas para que acentos e outros caracteres especiais apareçam diretamente no arquivo de origem,
deve-se dizer ao \TeX{} usar uma codificação diferente.


\LaTeX{} suporta a composição de caracteres especiais.
Isto é conveniente se o seu teclado não tiver alguns acentos desejados e outros diacríticos.

Os seguintes acentos podem ser colocados em letras.
Embora a letra 'o' seja usada na maioria dos exemplos, os acentos podem ser colocados em qualquer letra.
Os acentos podem até ser colocados acima de uma letra “ausente”; por exemplo, \verb|\ ~ {}| produz um til sobre um espaço em branco.

Para que a acentuação ocorra de forma automática deve-se usar os seguintes pacotes:

\begin{verbatim}
\usepackage[brazilian]{babel}
\usepackage[utf8]{inputenc}
\usepackage[T1]{fontenc}
\end{verbatim}

LaTeX tem muitos símbolos à sua disposição.
A maioria deles está no domínio matemático e os capítulos posteriores abordarão como obter acesso a eles.
Para os símbolos de texto mais comuns, use os seguintes comandos:

\begin{quadro}[!htb]
    \centering
    \caption{Exemplo Simbolos Especiais.\label{qua:quadro-simbolos-especiais}}
        \begin{tabular}{|l|l|l|}
            \hline
                \multicolumn{1}{|c|}{\textbf{Comando em \LaTeX{}}} & \multicolumn{1}{c|}{\textbf{Exemplo}} \\ \hline
                \verb|\$|                                       & \$                                     \\ \hline
                \verb|\&|                                       & \&                                      \\ \hline
                \verb|\%|                                       & \%                                       \\ \hline
        \end{tabular}
\end{quadro}