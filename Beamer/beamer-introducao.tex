\documentclass{beamer}

% Para justificar o texto
\usepackage{ragged2e}

\usepackage{verbatim}
\usepackage[brazilian]{babel}
\usepackage[T1]{fontenc}
\usepackage[utf8]{inputenc}

\mode<presentation>

\usetheme{Madrid}      % or try Darmstadt, Madrid, Warsaw, ...
\usecolortheme{default} % or try albatross, beaver, crane, ...
\usefonttheme{default}  % or try serif, structurebold, ...
\setbeamertemplate{navigation symbols}{}
\setbeamertemplate{caption}[numbered]

\title{Como criar Slides com \LaTeX}
\author{Jessica Dagostini; Marcos Lima}
\date{Junho de 2018}

\begin{document}

\begin{frame}
\titlepage
\end{frame}
	
\section{Introdução}
\begin{frame}{Introdução}
\begin{block}{O que é a classe \textbf{Beamer}?}
\justifying
O Beamer é uma classe do LATEX para criação de apresentações no formato PDF, o que as torna altamente portáveis. 
Sua estrutura é a mesma do LATEX, com algumas características específicas do Beamer.
É possível desenvolver apresentações dinâmicas com sobreposições e transiçcões animadas entre os quadros.
\end{block}
\end{frame}
    
\section{Primeiros Passos}
\begin{frame}[fragile]{Primeiros Passos}
\begin{block}{Classe Beamer}
Para utilizar a classe Beamer, assim como é feito com qualquer outra classe do LATEX, deve-se
declará-la da seguinte forma:
\begin{verbatim}
\documentclass{beamer}
\end{verbatim}
\end{block}
\end{frame}

\subsection{Estrutura}
\begin{frame}{Estrutura}
\begin{itemize}
\item A apresentação pode ser dividida em seções, subseções e quadros.
\item Uma apresentação geralmente começa com uma página de título, seguida pelo sumário e, então, seu conteúdo.
\end{itemize}
\end{frame}

\begin{frame}[fragile]{Estrutura}
\begin{block}{Exemplo}
\begin{verbatim}
\title[título curto]{título completo}
\author[nome abreviado]{nome completo}
\institute{nome da instituição à qual o apresentador está vinculado}
\date{\today}
\end{verbatim}
\end{block}
\end{frame}

\begin{frame}[fragile]{Estrutura}
\begin{block}{Gerar a Capa!}
\begin{verbatim}
%\begin{frame}
\titlepage
%\end{frame}
\end{verbatim}
\end{block}
\end{frame}

\section{Criando Slides}
\begin{frame}[fragile]{Criando Slides}
\begin{block}{Exemplo}
\begin{verbatim}
\section{Introdução}
\begin{frame}
...Texto...
%\end{frame}
\end{verbatim}
\end{block}
\end{frame}

\subsection{Criando Blocos de Textos}
\begin{frame}[fragile]{Criando Blocos de Textos}
\begin{block}{Exemplo}
\begin{verbatim}
\begin{block}{título do bloco}
conteúdo do bloco
\end{block}
\end{verbatim}
\end{block}
\end{frame}

\section{Efeitos}
\begin{frame}[fragile]{Efeitos}
\begin{table}[]
\centering
\caption{Efeitos nos Slides}
\label{my-table1}
\begin{tabular}{|l|l|}
\hline
\multicolumn{1}{|c|}{\textbf{Comando}} & \multicolumn{1}{c|}{\textbf{Efeito}}      \\ \hline
\textbackslash{}transblindshorizontal  &  Cortinas horizontas se afastando          \\ \hline
\textbackslash{}transblindsvertical    &  Cortinas verticais se afastando            \\ \hline
\textbackslash{}transboxin             &  Movimento das bordas ao centro              \\ \hline
\textbackslash{}transboxout            &  Movimento do centro às bordas       		   \\ \hline
\textbackslash{}transdissolve          &  Dissolver devagar o conteúdo anterior         \\ \hline
\textbackslash{}transglitter           &  Efeito Glitter numa direção específica         \\ \hline
\textbackslash{}transslipverticalin    &  O conteúdo entra em duas linhas verticais       \\ \hline
\textbackslash{}transslipverticalout   &  O conteúdo sai em duas linhas verticais          \\ \hline
\textbackslash{}transhorizontalin      &  O conteúdo Entra em duas linhas horizontais       \\ \hline
\textbackslash{}transhorizontalout     &  O conteúdo sai em duas linhas horizontais          \\ \hline
\end{tabular}
\end{table}
\end{frame}

\section{Temas}
\begin{frame}[fragile]{Temas}
\begin{block}{Exemplo}
\begin{verbatim}
\usetheme{nome do tema} % Darmstadt, Madrid, Warsaw, ...
\usecolortheme{default} % albatross, beaver, crane, ...
\usefonttheme{default}  % serif, structurebold, ...
\setbeamertemplate{navigation symbols}{}
\setbeamertemplate{caption}[numbered]
\end{verbatim}
\end{block}
\href{https://hartwork.org/beamer-theme-matrix/}{\beamergotobutton{Exemplo de Temas}}
\end{frame}

\section{FIM!}
\begin{frame}{FIM!}
Obrigado :D
\end{frame}
\end{document}