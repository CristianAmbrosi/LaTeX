% Arquivo para resolução do exercício ...(colocar número do exercício)

\chapter{INTRODUÇÃO}
\begin{large}
O sistema de posicionamento global
\end{large}
\foreignlanguage{english}{(positioning system)} %aumente a fonte para 14pts 
, mais conhecido pela sigla GPS (em inglês global positioning system) %Colocar trecho em inglês [positioning system]
é um sistema de posicionamento por satélite que fornece a um aparelho receptor móvel a sua posição, assim como informação horária,
sob quaisquer condições atmosféricas, a qualquer momento e em qualquer lugar na Terra, desde que
o receptor se encontre no campo de visão de três satélites GPS

Todos os satélites são controlados pelas estações terrestres de
gerenciamento. Existe uma que é a \textit{master} %Colocar [master] em itálico
, localizada no Colorado (Estados Unidos), que, com o auxílio de cinco estações de gerenciamento
espalhadas pelo planeta, monitoram o desempenho total do sistema, corrigindo
as posições dos satélites e reprogramando o sistema com o padrão
necessário. Após o processamento de todos esses dados, as correções e
sinais de controle são transferidos de volta para os satélites.

Cada um dos satélites do \textbf{GPS} %Colocar [GPS] em negrito
transmite por rádio um padrão fixado,
que é recebido por um receptor na Terra (segmento do usuário), funcionando
como um cronômetro extremamente acurado. O receptor mede a diferença
entre o tempo que o padrão é recebido e o tempo que foi emitido. Essa
diferença, não mais do que um décimo de segundo, permite que o receptor
calcule a distância ao satélite emissor multiplicando-se a 
velocidade\footnote{aproximadamente 2,99792458.108 m/s – a velocidade da luz} do sinal %colocar nota de rodapé em [velocidade] texto = ( aproximadamente 2,99792458.108 m/s – a velocidade da luz )
pelo tempo que o sinal de rádio levou do satélite ao receptor.

\begin{figure}[!htb]
    \centering
    \caption{Exemplo de Figura}
    \includegraphics[width=0.3\textwidth]{ltx_paper}
    \fonte{\cite{IRL2014}}
    \label{fig:figura-exemplo1}
\end{figure}

% Colocar esse parágrafo na outra página
\newpage Em \cite{Barbosa2004} 24 de março 2009 foi lançado o primeiro satélite GPS equipado com uma amostra de hardware funcionando em frequência l5.
\cite{Barbosa2004} Entre outras novidades, este satélite será o primeiro a emitir o sinal GPS numa frequência de 1176.45 MHz (1.2 GHz).

Em Geral isso é uma vantagem pois:
% Colocar frases abaixo em itens não numerados
\begin{itemize}
    \item Melhora a estrutura do sinal para melhor desempenho.
    \item Transmissão superior ao do L1 e L2 sinal.
\end{itemize}
% Centralize este parágrafo
\begin{center}
Lorem ipsum dolor sit amet, consectetur adipiscing elit.
\end{center}

\begin{flushleft}
% Alinhe à esquerda
Lorem ipsum dolor sit amet, consectetur adipiscing $E=mc^2$ elit.
\end{flushleft}

\begin{flushright}
% Alinhe à direita
Lorem ipsum dolor sit amet, consectetur adipiscing elit \(E=mc^2\).
\end{flushright}

\begin{table}[!htb]
    \centering
    \caption[Resultado dos testes]{Resultado dos testes.
    \label{tab:tabela-exemplo1}}
    \begin{tabular}{rrrrr}
        \toprule
            & Valores 1 & Valores 2 & Valores 3 & Valores 4 \\
        \midrule
            Caso 1 & 0,86 & 0,77 & 0,81 & 163 \\
            Caso 2 & 0,19 & 0,74 & 0,25 & 180 \\
            Caso 3 & 1,00 & 1,00 & 1,00 & 170 \\
        \bottomrule
    \end{tabular}
    \fonte{\cite{Barbosa2004}}
\end{table}

\begin{table}[!htb]
    \centering
    \caption[Faixa Estária de Estudantes do Curso.]{Faixa Estária de Estudantes do Curso.
    \label{tab:tabela-exemplo1}}
    \begin{tabular}{cc}
        \toprule
            Faixa Etária & Frequência Absoluta \\
        \midrule
            19 $\vdash$ 21 & 6 \\
            21 $\vdash$ 23 & 14 \\
            23 $\vdash$ 25 & 1 \\
            25 $\vdash$ 27 & 25 \\
        \bottomrule
    \end{tabular}
    \fonte{EU}
\end{table}

    \begin{math}
        E=mc^2
    \end{math}

    \begin{algorithm}[H]
    \Entrada{o proprio texto}
    \Saida{como escrever algoritmos com \LaTeX2e }
    \Inicio{
        inicializa\c{c}\~ao\;
        \Repita{fim do texto}{
            leia o atual\;
            \Se{entendeu}{
                vá para o próximo\;
                próximo se torna o atual\;}
            \Senao{volte ao início da seção\;}
        }
    }
    \caption{Como escrever algoritmos no \LaTeX2e}
    \end{algorithm}
