% ESTRUTURA E COMANDOS--------------------------------------------------------

\chapter{ESTRUTURA E COMANDOS}
\label{chap:estruturaecomandos}

Um documento em \LaTeX{} começa pelo comando \verb|\documentclass[opcionais]{classe}|.
Abaixo desse comando são declarados os pacotes e outras configurações do documento.

Os comandos \verb|\begin{document}| e \verb|\end{document}| marcam o início do ambiente onde por exemplo,
as seções, subseções e o texto do documento serão inseridos. 
O seguinte trecho de código exemplifica a estrutura básica de um documento em \LaTeX.

\begin{verbatim}
\documentclass[opcionais]{classe}

    % declarações

\begin{document}

    % documento

\end{document}
\end{verbatim}

As principais\footnote{pode se encontrar encontrar outros parâmetros para classes e opções nesse link:
\href{https://en.wikibooks.org/wiki/LaTeX/Document_Structure}{\textcolor{blue}{Classes e Opções}}} 
classes que podem ser passadas para o comando \verb|\documentclass| são:\\

\begin{itemize}
    \item \{article\} - para produzir artigos curtos;
    \item \{report\} - para produzir artigos mais longos, relatórios e monografias;
    \item \{book\} - para produzir livros.\\
\end{itemize}

Nas opções é possível utilizar os seguintes parâmetros:\\

\begin{itemize}
    \item 11pt - fonte de 11 pontos;
    \item 12pt - fonte de 12 pontos;
    \item twoside - imprime em ambos os lados da folha;
    \item twocolumn - produz o documento formatado em duas colunas;
\end{itemize}

% ESTRUTURA E COMANDOS - Trabalhando com Projetos Grandes--------------------

\section{Trabalhando com Projetos Grandes}

O comando \verb|\include{nome_do_arquivo}| inclui um arquivo ao documento principal.
Este comando deve ser usado quando existem capítulos com quantidades maiores de conteúdo.

% ESTRUTURA E COMANDOS - Parâmetros Obrigatórios e Opcionais-----------------

\section{Parâmetros Obrigatórios e Opcionais}

Os comandos em LaTeX são precedidos por \textbackslash e seguidos por [ ] e/ou \{ \}, onde:\\ 

[ ] Parâmetros opcionais.

\{ \} Parâmetros Obrigatórios.

% ESTRUTURA E COMANDOS - Comentários-----------------------------------------

\section{Comentários}

Comentários em um documento \TeX{} são ignorados pelo compilador.
Para utilizar comentários ao longo do texto utilize \%