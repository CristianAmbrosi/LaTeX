% INTRODUÇÃO-------------------------------------------------------------------

\chapter{INTRODUÇÃO}
\label{chap:introducao}

\LaTeX \xspace é um conjunto de macros de alto nível para \TeX \xspace que torna mais fácil e rápida a produção de todo o tipo de documentos como, por exemplo, livros, relatórios e artigos.

O objetivo do \LaTeX \xspace é que o autor se possa distanciar da apresentação visual do trabalho e assim se concentrar no seu conteúdo. Possui formas de lidar com bibliografias, citações, formatos de páginas, referências e tudo mais que não seja relacionado com conteúdo do documento em si.

\section{Histórico}
\label{sec:Histórico}

Em 1978 Donald E. Knuth começou a desenvolver uma linguagem cujo objetivo era permitir a qualquer um formatar textos com muitas equações e com alta qualidade de saída, chamada de \TeX. Em 1985 Leisle Lamport desenvolveu um conjunto de macros denominado \LaTeX \xspace, que simplifica o uso da linguagem \TeX \xspace. Atualmente este projeto é mantido e desenvolvido pelo \LaTeX3 \xspace Project

O som final dos nomes \TeX \xspace e \LaTeX \xspace deve ser pronunciado como se fosse um “K”.

\section{Instalação}
\label{sec:Instalação}

\LaTeX é um software livre e gratuito, é possível instalar nos  principais sistemas operacionais modernos como: Windows 7, 8, 8.1 e 10; Mac OS e várias distribuições Linux.

\subsection{Windows}

MiKTeX é uma distribuição TeX / LaTeX para o Microsoft Windows. Baixe o instalador pelo link oficial: \href{https://miktex.org/download}{\textcolor{blue}{Download MiKTeX}}

A instalação do MiKTeX é simples, basicamente é só clicar no “Avançar”. Caso houver problemas na instalação o seguinte vídeo poderá servir de ajuda: \href{https://www.youtube.com/watch?v=4udFXbqtayE&list=LLQVoeslEpxQJ0UavpXUEkq}{\textcolor{blue}{Vídeo Instalação MiKTeX}}. Junto com a instalação do MiKTeX o editor TeXworks é instalado. Existem outros editores como o Texmaker. O Texmaker é mais simples de usar e é o editor de código aberto mais popular entre a comunidade LaTeX.

O Texmaker é um editor LaTeX livre, moderno e multiplataforma para sistemas Linux, Macosx e Windows que integra muitas ferramentas necessárias para desenvolver documentos com o LaTeX. Baixe o instalador pelo link oficial: \href{http://www.xm1math.net/texmaker/download.html}{\textcolor{blue}{Download TeXMaker}}

\subsection{Ubuntu}

O TeX Live é uma distribuição para produção de documentos \TeX \xspace. Para instalar no ubuntu 16.04 digite o seguinte comando: \emph{sudo apt-get install texlive-full}

Apos a instalação do TeX Live pode-se obtar por instalar um editor específico para LaTeX o Texmaker, para instalar digite o seguinte comando: \emph{sudo apt-get install texmaker}